%%%%%%%%%%%%%%%%%%%%%%%%%%%%%%%%%%%%%%%%%
% Beamer Presentation
% LaTeX Template
% Version 1.0 (10/11/12)
%
% This template has been downloaded from:
% http://www.LaTeXTemplates.com
%
% License:
% CC BY-NC-SA 3.0 (http://creativecommons.org/licenses/by-nc-sa/3.0/)
%
%%%%%%%%%%%%%%%%%%%%%%%%%%%%%%%%%%%%%%%%%

%----------------------------------------------------------------------------------------
%	PACKAGES AND THEMES
%----------------------------------------------------------------------------------------

\documentclass{beamer}

\mode<presentation> {

% The Beamer class comes with a number of default slide themes
% which change the colors and layouts of slides. Below this is a list
% of all the themes, uncomment each in turn to see what they look like.

%\usetheme{default}
%\usetheme{AnnArbor}
%\usetheme{Antibes}
%\usetheme{Bergen}
%\usetheme{Berkeley}
%\usetheme{Berlin}
%\usetheme{Boadilla}
\usetheme{CambridgeUS}
%\usetheme{Copenhagen}
%\usetheme{Darmstadt}
%\usetheme{Dresden}
%\usetheme{Frankfurt}
%\usetheme{Goettingen}
%\usetheme{Hannover}
%\usetheme{Ilmenau}
%\usetheme{JuanLesPins}
%\usetheme{Luebeck}
%\usetheme{Madrid}
%\usetheme{Malmoe}
%\usetheme{Marburg}
%\usetheme{Montpellier}
%\usetheme{PaloAlto}
%\usetheme{Pittsburgh}
%\usetheme{Rochester}
%\usetheme{Singapore}
%\usetheme{Szeged}
%\usetheme{Warsaw}

% As well as themes, the Beamer class has a number of color themes
% for any slide theme. Uncomment each of these in turn to see how it
% changes the colors of your current slide theme.

%\usecolortheme{albatross}
%\usecolortheme{beaver}
%\usecolortheme{beetle}
%\usecolortheme{crane}
%\usecolortheme{dolphin}
%\usecolortheme{dove}
%\usecolortheme{fly}
%\usecolortheme{lily}
%\usecolortheme{orchid}
%\usecolortheme{rose}
%\usecolortheme{seagull}
%\usecolortheme{seahorse}
%\usecolortheme{whale}
%\usecolortheme{wolverine}

%\setbeamertemplate{footline} % To remove the footer line in all slides uncomment this line
%\setbeamertemplate{footline}[page number] % To replace the footer line in all slides with a simple slide count uncomment this line

%\setbeamertemplate{navigation symbols}{} % To remove the navigation symbols from the bottom of all slides uncomment this line
}

\usepackage[orientation=landscape,size=custom,width=16,height=9,scale=0.4]{beamerposter}
\usepackage[utf8x]{inputenc}
\usepackage{algorithmicx}
\usepackage{algorithm}% http://ctan.org/pkg/algorithms
\usepackage{algpseudocode}% http://ctan.org/pkg/algorithmicx
\usepackage{amsfonts}
\usepackage{amsmath,amssymb,amstext,mathabx}
\usepackage{amsmath}
\usepackage{amsthm,lipsum}
\usepackage{booktabs} % Allows the use of \toprule, \midrule and \bottomrule in tables
\usepackage{graphicx} % Allows including images
\usepackage{hyperref}
\usepackage{listings}
\usepackage{subcaption}
\setbeamersize{text margin left=5em}  % <- like this
\setbeamersize{text margin right=5em} % <- like this

%My Maths operators
\DeclareMathOperator{\opt}{opt}
\DeclareMathOperator{\web}{web}
\DeclareMathOperator{\PO}{PO}
\DeclareMathOperator{\DM}{DM}
\DeclareMathOperator{\MT}{MT}
\DeclareMathOperator{\mtx}{mtx}
\DeclareMathOperator{\nrows}{nrows}
\DeclareMathOperator{\ncols}{ncols}
\DeclareMathOperator{\potpos}{Potential Positions}
\DeclareMathOperator{\nodf}{nodf}
\DeclareMathOperator{\NODF}{NODF}
\DeclareMathOperator{\neighbor}{neighbor}
\DeclareMathOperator{\random}{random}
\DeclareMathOperator{\cost}{cost}
\DeclareMathOperator{\hillclimb}{Hillclimb}
\DeclareMathOperator{\Olap}{O}
\DeclareMathOperator{\DF}{DF}
\DeclareMathOperator{\row}{row}
\DeclareMathOperator{\col}{col}
\DeclareMathOperator{\Fmin}{Fmin}
\newcommand{\N}{\mathbb{N}}
\newcommand{\R}{\mathbb{R}}

\setbeamertemplate{navigation symbols}{}
\beamertemplatenavigationsymbolsempty

%\newenvironment{sysmatrix}[1]
% {\left(\begin{array}{@{}#1@{}}}
% {\end{array}\right)}
%\newcommand{\ro}[1]{%
%  \xrightarrow{\mathmakebox[\rowidth]{#1}}%
%}
%\newlength{\rowidth}% row operation width
%\AtBeginDocument{\setlength{\rowidth}{3em}}

%----------------------------------------------------------------------------------------
%	TITLE PAGE
%----------------------------------------------------------------------------------------

\title[T-Shirt Problem]{Optimising T-Shirt distributiuon} % The short title appears at the bottom of every slide, the full title is only on the title page

\author{A. Puiu, Y. Tian, J. Dyer, C. Hoeppke} % Your name
\institute[University of Oxford] % Your institution as it will appear on the bottom of every slide, may be shorthand to save space
{
%University of Oxford\\
%\medskip
\textit{Department for Mathematics} % Your email address
\date{\today} % Date, can be changed to a custom date
}
%\date{01/10/2018} % Date, can be changed to a custom date
\begin{document}


\begin{frame}
%Probably the titlepage will not be needed!
\titlepage % Print the title page as the first slide
\end{frame}

\section{Description of the Problem}
% Describe the problem
% Problem description section
\begin{frame}{Problem description}
    \begin{block}{T-Shirt distribution problem}
        You are a T-Shirt distributor and want to optimise your packaging
        process when sending boxes of T-Shirts to stores.
    \end{block}
    \only<1>{
        \begin{center}
            \includegraphics[width=0.75\textwidth]{./figures/problem-overview-crop.pdf}
        \end{center}
    }
    \only<2>{
    \begin{itemize}
        \item It is expensive to train your staff on packing T-Shirts into many
        different types of boxes.
        \item Each store requires a
        certain number of T-Shirts of given sizes.
        \item Stores will only accept receiving more T-Shirts than ordered in the case of
        medium and large Shirts in colors black, blue or red.
        \item Over-stocking is bad, but under-stocking is a lot (say $10$ times) worse.
        \item You are supplied with boxes that fit 4, 6, 8
        or 10 T-Shirts and can not ship partially filled boxes.
        \item You offer a total of $24$ different T-Shirts.\\
            \hfill(Six different sizes offered in four different colors)
    \end{itemize}}
\end{frame}

\begin{frame}{Optimisation problem}
    Variables in the optimisation problem
    \begin{align*}
        \text{Orders: }& \text{o}_i \in \N_{\geq 0}^{24}\\
        \text{Boxes: }& \text{b}_j \in \N_{\geq 0}^{24}, \text{ with }
        \text{b}_j^T 1 \in \{4, 6, 8, 10\}, \text{ for all } j \in 1,...,M\\
        \text{Delivery plan: }& \alpha_{(i,j)}\in \N_{\geq 0}~ \forall i, j\\
        \text{Deliveries: }& \text{s}_i =
        \sum_{j = 1,\dots,M}\alpha_{(i,j)}\text{b}_j~\text{ for all } j \in 1,\dots,58\\
    \end{align*}
    Where $\alpha_{i,j}$ is the number of boxes of type $j$ that store $i$ will
    receive and $M$ is the total number of different box-types used.
\end{frame}

\begin{frame}{Formulation as an integer optimisation problem}
    \begin{align*}
        \text{minimize: } & \underbrace{\beta M}_{\text{Penalise number of
        boxes}} + \sum_{i = 1}^{58}
        \underbrace{\mathbbm{1}_{\text{s}_i > \text{o}_i} (\text{s}_i -
        \text{o}_i)}_{\text{Penalise overstocking}} +
        \underbrace{
            10 \cdot \mathbbm{1}_{\text{s}_i < \text{o}_i}(\text{o}_i -
            \text{s}_i)}_{\text{Penalise understocking more}}\\
        \text{subject to:}&\\
        &\text{M} \in \N_{\geq 1}\\
        &\text{b}_j \in \N_{\geq 0}^{24}~\forall~j = 1,\dots,M\\
        &\alpha_{(i,j)} \in \N_{\geq 0}^{24}~\forall~i=1,\dots,58,~j=1,\dots,M\\
        &\text{s}_i = \sum_{j = 1,\dots,M}\alpha_{(i,j)}\text{b}_j \leq
        \text{o}_i^{\text{limit}}~\forall~i = 1,\dots,58\\
    \end{align*}
    \begin{itemize}
        \item Note: Optimal solution depends on penalty parameter $\beta > 0$.\\
        \item In general integer optimisation problems are NP-Hard.\\
    \end{itemize}

\end{frame}


% Problem description section
\begin{frame}{Problem description}
    \begin{block}{T-Shirt distribution problem}
        You are a T-Shirt distributor and want to optimise your packaging
        process when sending boxes of T-Shirts to stores.
        \begin{itemize}
            \item Each store requires a
            certain number of T-Shirts of given sizes.
            \item It is expensive to train your staff on packing T-Shirts into many
            different types of boxes. You are supplied with boxes that fit 4, 6, 8
            or 10 T-Shirts and can not ship partially filled boxes.
            \item You offer a total of $24$ different T-Shirts (Six different sizes
            offered in four different colors).
        \end{itemize}
    \end{block}
    Variables:
    \begin{minipage}[h]{0.45\textwidth}
        Orders\\
        \begin{align*}
            \text{order}_i \in \N_{\geq 0}^{24}
        \end{align*}
    \end{minipage}
    \begin{minipage}[h]{0.45\textwidth}
        Boxes\\
        \begin{align*}
            \text{box}_i \in \N_{\geq 0}^{24}, \text{ with } \sum_{j = 1}^{24}
            \text{box}_i^{(j)} \in {4, 6, 8, 10}
        \end{align*}
    \end{minipage}
\end{frame}

\subsection{Brute force approach}
% Use this the clarify why the problem is difficult.
\begin{frame}
\begin{algorithm}[H]
	\begin{algorithmic}[1] % The number tells where the line numbering should start
		\State $N_{max}$ = maximum possible number of box types
		\ForAll{$N \in \lbrace{1, 2, \dots, N_{max}\rbrace}$}
			\ForAll{possible combinations of N box types}
				\State compute penalty given $N$ and total over- and under-stock
			\EndFor
		\EndFor
	\end{algorithmic}
	\caption{Brute-force algorithm}
\end{algorithm}
However, $N_{max} \sim 10^{11}$, so $\sim 2^{10^{11}} - 1$ combinations of box types to consider\\
$\Rightarrow$ brute force approach unfeasible
\end{frame}



\section{Pattern detection}
% Section that contains the pattern detection
\begin{frame}{Initial Analysis}
	The Data:
	\begin{itemize}
		\item 58 stodes with 4 colours and 6 sizes each.
		\item only M and L for black, blue and red can be overstacked.
	\end{itemize}
	Fundamental Problem:
	\begin{itemize}
		\item Boxes require even number of shirt, but some stores order odd numbers $\Rightarrow$ overstocking is unavoidable.
	\end{itemize}
	Try to find further structure:
	\begin{itemize}
		\item min,average,mode,sum $Rightarrow$ subtract the mode $Rightarrow$ subtract the min $\Rightarrow$ pattern inside each colour
	\end{itemize}
\end{frame}
\begin{frame}{Initial Analysis ct'd}

\begin{equation}
     \begin{bmatrix}
     1&2&2&3&2&1\\
  0&0&2&0&0&0\\
  0&0&0&2&0&0\\
  0&0&0&0&2&0
     \end{bmatrix}
\end{equation}
\begin{equation}
\color{Blue}
    \begin{bmatrix}
        1&2&2&3&3&1\\
        0&0&0&2&0&0\\
        0&0&1&0&1&0\\
        0&0&1&1&0&0\\
    \end{bmatrix}
\end{equation}
\begin{equation}
\color{BrickRed}
    \begin{bmatrix}
         1&2&3&3&3&1\\
         0&0&0&1&2&0\\
         0&0&0&3&0&0\\
         0&1&1&0&0&0\\
    \end{bmatrix}
\end{equation}
\begin{equation}
\color{OliveGreen}
    \begin{bmatrix}
         0&0&0&0&0&0\\
         1&1&1&2&1&1\\
         1&1&2&2&1&1\\
         1&1&2&2&2&1\\

    \end{bmatrix}
\end{equation}
\end{frame}

\begin{frame}{Colour Based Optimisation}
	\begin{itemize}
		\item We form boxes for each colour individually
	\end{itemize}
	\begin{equation*}

     \begin{bmatrix}
     1&2&2&3&2&1\\
  0&0&2&0&0&0\\
  0&0&0&2&0&0\\
  0&0&0&0&2&0
     \end{bmatrix} \Rightarrow
     \begin{bmatrix}
     1&2&2&2&2&1\\
	     0&0&2&1+\Color{Red}1&0&0\\
	     0&0&0&3+\Color{Red}1&0&0\\
	     0&0&0&1+\Color{Red}1&2&0
     \end{bmatrix}


	\end{equation*}
	\begin{equation*}
\color{Blue}
    \begin{bmatrix}
        1&2&2&3&3&1\\
        0&0&0&2&0&0\\
        0&0&1&0&1&0\\
        0&0&1&1&0&0\\
    \end{bmatrix}\Rightarrow
\color{Blue}
    \begin{bmatrix}
        1&2&2&2&2&1\\
        0&0&0&3&1&0\\
        0&0&1&1&2&0\\
        0&0&1&2&1&0\\
    \end{bmatrix}
	\end{equation*}

\end{frame}
\begin{frame}
	\begin{equation}
	\color{BrickRed}
    \begin{bmatrix}
         1&2&3&3&3&1\\
         0&0&0&1&2&0\\
         0&0&0&3&0&0\\
         0&1&1&0&0&0\\
    \end{bmatrix}\Rightarrow \begin{bmatrix}
	    1&2&2&2&2&1\\
         0&0&1&2&3&0\\
         0&0&1&4&1&0\\
	    0&1&2&1&1+\Color{Grey}1&0\\
    \end{bmatrix}
\end{equation}
\begin{equation}
\color{OliveGreen}
    \begin{bmatrix}
         0&0&0&0&0&0\\
         1&1&1&2&1&1\\
         1&1&2&2&1&1\\
         1&1&2&2&2&1\\

    \end{bmatrix}\Rightarrow \begin{bmatrix}
         1&1&1&2&1&1\\
         1&1&2&2&1&1\\
         1&1&2&2&2&1\\

    \end{bmatrix}
\end{equation}
$\Rightarrow$ This gives $4+5+4+3=16$ types, but we can do better for blue!
	\begin{frame}
	\begin{equation}

    \begin{bmatrix}
        1&2&2&2&2&1\\
        0&0&0&3&1&0\\
        0&0&1&1&2&0\\
        0&0&1&2&1&0\\
    \end{bmatrix} \Rightarrow \begin{bmatrix}
	    1& 0& 0& 1& 3& 1\\
	    0& 1& 2& 1& 0& 0\\
	    0& 1& 1& 2& 0& 0\\
	    0& 1& 1& 1& 1& 0 \\
    \end{bmatrix}
	\end{equation}
    $\Rightarrow$ Now $16-1=15$ types.
\end{frame}

	\begin{frame}{Cross Colour Optimisation}
	\begin{itemize}
	\item full rank now, to further reduce the types of boxes, we need to consider the combination of them
	\item to avoid the mess of combination for each store, we consider combine the box 1 in black and box 1 in blue, as they are what everybody needs
	\end{itemize}
    \begin{equation}
        \begin{bmatrix}
            0& 0& 0& 0& 2& 1\\
            1& 2& 4& 3& 0& 0 \\
             1& 2& 2& 5& 0& 0\\
             1& 2& 2& 3& 2& 0
        \end{bmatrix}
    \end{equation}
			$\Rightarrow$ then combine the box 1 in black with box 1 in blue with extra 1 in black L (acceptable overload)
now $15 – 1 = 14$ types!
	\end{frame}





\end{frame}


\subsection{Initial look at the data}
% Introduce the data-set
% Describe problem in the dataset -> Some stores want odd numbers of Shirts
% Detect the color-related patterns
\begin{frame}{Initial Look at the Data}
\begin{block}{our data}
	58 stores with 4 colour and each has 6 sizes with only M and L in black, blue and red can be overstock 
\end{block}
\begin{table}[htbp]
	\begin{center}
		\begin{tabular}{|r|r|r|r|r|r|r|r|r|r|r|r|r|}
			\hline
			\multicolumn{1}{|l|}{} & \multicolumn{6}{c|}{Black} &  \multicolumn{6}{c|}{Blue} \\ \hline
			\multicolumn{1}{|l|}{No. Store} & \multicolumn{1}{l|}{XS} & \multicolumn{1}{l|}{S} & \multicolumn{1}{l|}{M} & \multicolumn{1}{l|}{L} & \multicolumn{1}{l|}{XL} & \multicolumn{1}{l|}{2XL} & \multicolumn{1}{l|}{XS} & \multicolumn{1}{l|}{S} & \multicolumn{1}{l|}{M} & \multicolumn{1}{l|}{L} & \multicolumn{1}{l|}{XL} & \multicolumn{1}{l|}{2XL} \\ \hline
			1 & 1 & 2 & 2 & 3 & 4 & 1 & 1 & 2 & 2 & 5 & 3 & 1 \\ \hline
			2 & 1 & 2 & 4 & 3 & 2 & 1 & 1 & 2 & 3 & 3 & 4 & 1 \\ \hline
			3 & 1 & 2 & 2 & 5 & 2 & 1 & 1 & 2 & 2 & 5 & 3 & 1 \\ \hline
			4 & 1 & 2 & 2 & 3 & 4 & 1 & 1 & 2 & 4 & 3 & 3 & 1 \\ \hline
			5 & 1 & 2 & 4 & 3 & 2 & 1 & 1 & 2 & 3 & 3 & 4 & 1 \\ \hline
			6 & 1 & 2 & 4 & 3 & 2 & 1 & 1 & 2 & 4 & 3 & 3 & 1 \\ \hline
		\end{tabular}
	\end{center}
	\caption{data - first 6 rows of black and blue}
	\label{}
\end{table}
\end{frame}

\begin{frame}{Initial Look at the Data}
 \begin{block}{fundamental problem} 
	even number for all types of boxes vs some stores have odd sums → overstock is unavoidable
\end{block}
\begin{block}{further structure}
	\begin{itemize}
		\item min, average, mode, sum → subtract the mode → subtract the min 
		\item small scale first → pattern inside each color
	\end{itemize}
\end{block}
\end{frame}

\begin{frame}{Initial Look at the Data}
\begin{minipage}{\textwidth}
\begin{table}[htbp]
	\begin{center}
		\begin{tabular}{|l|r|r|r|r|r|r|l|r|r|r|r|r|r|}
			\hline
			& \multicolumn{6}{|c|}{Black} & \multicolumn{1}{|c|}{} & \multicolumn{6}{|c|}{Blue} \\ \hline
			Size & \multicolumn{1}{l|}{XS} & \multicolumn{1}{l|}{S} & \multicolumn{1}{l|}{M} & \multicolumn{1}{l|}{L} & \multicolumn{1}{l|}{XL} & \multicolumn{1}{l|}{XXL}& \multicolumn{1}{|c|}{} & \multicolumn{1}{l|}{XS} & \multicolumn{1}{l|}{S} &  \multicolumn{1}{l|}{M} & \multicolumn{1}{l|}{L} & \multicolumn{1}{l|}{XL} & \multicolumn{1}{l|}{XXL} \\ \hline 
			min  & 1 & 2 & 2 & 3 & 2 & 1 &  & 1 & 2 & 2 & 3 & 3 & 1 \\ \hline
			& 0 & 0 & 2 & 0 & 0 & 0 &  & 0 & 0 & 0 & 2 & 0 & 0 \\ \hline
			& 0 & 0 & 0 & 2 & 0 & 0 &  & 0 & 0 & 1 & 0 & 1 & 0 \\ \hline
			& 0 & 0 & 0 & 0 & 2 & 0 &  & 0 & 0 & 2 & 0 & 0 & 0 \\ \hline
			& \multicolumn{1}{l|}{} & \multicolumn{1}{l|}{} & \multicolumn{1}{l|}{} & \multicolumn{1}{l|}{} & \multicolumn{1}{l|}{} & \multicolumn{1}{l|}{} &  & 0 & 0 & 1 & 1 & 0 & 0 \\ \hline
		\end{tabular}
	\end{center}
	\caption{patterns in black(left) and blue(right)}
	\label{}
\end{table}\vspace{-3ex}
\end{minipage}
\begin{minipage}{\textwidth}
	
	\begin{table}[htbp]
		\begin{center}
			\begin{tabular}{|l|r|r|r|r|r|r|l|r|r|r|r|r|r|}
				\hline
				\multicolumn{1}{|c|}{} & \multicolumn{6}{c|}{Red} & \multicolumn{1}{|c|}{} &  \multicolumn{6}{c|}{Green}  \\ \hline
				Size & \multicolumn{1}{l|}{XS} & \multicolumn{1}{l|}{S} & \multicolumn{1}{l|}{M} & \multicolumn{1}{l|}{L} & \multicolumn{1}{l|}{XL} & \multicolumn{1}{l|}{XXL}& \multicolumn{1}{|c|}{} & \multicolumn{1}{l|}{XS} & \multicolumn{1}{l|}{S} & \multicolumn{1}{l|}{M} & \multicolumn{1}{l|}{L} & \multicolumn{1}{l|}{XL} & \multicolumn{1}{l|}{XXL} \\ \hline
				min & 1 & 2 & 3 & 3 & 3 & 1 &  & 0 & 0 & 0 & 0 & 0 & 0 \\ \hline
				& 0 & 0 & 0 & 1 & 2 & 0 &  & 1 & 1 & 1 & 3 & 1 & 1 \\ \hline
				& 0 & 0 & 0 & 3 & 0 & 0 &  & 1 & 1 & 2 & 2 & 1 & 1 \\ \hline
				& 0 & 1 & 1 & 0 & 0 & 0 &  & 1 & 1 & 1 & 2 & 2 & 1 \\ \hline
			\end{tabular}
		\end{center}
		\caption{patterns in red (left) and green(right)}
		\label{}
	\end{table}
	
\end{minipage}
\end{frame}

% Introduce the data-set
% Describe problem in the dataset -> Some stores want odd numbers of Shirts
% Detect the color-related patterns

\subsection{Optimisation of box-types based on colors}
\include{color-optimisation.tex}


\subsection{Cross-color optimisation}
\include{cross-color-optimisation.tex}

\subsection{Final results}
\begin{frame}{Final Results}
    \begin{figure}
        \centering
        \includegraphics[width=8cm]{./figures/boxtypes.png}        \label{fig:my_label}
    \end{figure}
\end{frame}


\end{document}
